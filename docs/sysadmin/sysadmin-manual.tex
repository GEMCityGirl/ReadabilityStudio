% Options for packages loaded elsewhere
\PassOptionsToPackage{unicode}{hyperref}
\PassOptionsToPackage{hyphens}{url}
\PassOptionsToPackage{dvipsnames,svgnames,x11names}{xcolor}
\PassOptionsToPackage{space}{xeCJK}
%
\documentclass[
  letterpaper,
]{LegrandOrangeBook}

%%%%%%%%%%%%%%%%%%%%%%%%%%%%%%%%%%%%%%%%%%%%%%%%%%%%%%%%%%%%%%%%%%%%%%%
%
% Settings for Legrand Orange Book document class and other decorations
%
%%%%%%%%%%%%%%%%%%%%%%%%%%%%%%%%%%%%%%%%%%%%%%%%%%%%%%%%%%%%%%%%%%%%%%%

  \definecolor{ocre}{RGB}{26, 69, 135}

  \chapterimage{non-generated/banner.jpg} % Chapter heading image
\chapterspaceabove{6.5cm} % Default whitespace from the top of the page to the chapter title on chapter pages
\chapterspacebelow{6.75cm} % Default amount of vertical whitespace from the top margin to the start of the text on chapter pages

% title page

  \definecolor{titlepagecolor}{RGB}{0, 128, 128}

% The following code is borrowed from: https://tex.stackexchange.com/a/86310/10898

\newcommand\titlepagedecoration{%
\begin{tikzpicture}[remember picture,overlay,shorten >= -10pt]

\coordinate (aux1) at ([yshift=-15pt]current page.north east);
\coordinate (aux2) at ([yshift=-410pt]current page.north east);
\coordinate (aux3) at ([xshift=-4.5cm]current page.north east);
\coordinate (aux4) at ([yshift=-150pt]current page.north east);

\begin{scope}[titlepagecolor!40,line width=12pt,rounded corners=12pt]
\draw
  (aux1) -- coordinate (a)
  ++(225:5) --
  ++(-45:5.1) coordinate (b);
\draw[shorten <= -10pt]
  (aux3) --
  (a) --
  (aux1);
\draw[opacity=0.6,titlepagecolor,shorten <= -10pt]
  (b) --
  ++(225:2.2) --
  ++(-45:2.2);
\end{scope}
\draw[titlepagecolor,line width=8pt,rounded corners=8pt,shorten <= -10pt]
  (aux4) --
  ++(225:0.8) --
  ++(-45:0.8);
\begin{scope}[titlepagecolor!70,line width=6pt,rounded corners=8pt]
\draw[shorten <= -10pt]
  (aux2) --
  ++(225:3) coordinate[pos=0.45] (c) --
  ++(-45:3.1);
\draw
  (aux2) --
  (c) --
  ++(135:2.5) --
  ++(45:2.5) --
  ++(-45:2.5) coordinate[pos=0.3] (d);   
\draw 
  (d) -- +(45:1);
\end{scope}
\end{tikzpicture}%
}

% add the title page decoration
\makeatletter
\def\@endpart{\titlepagedecoration\vfil\newpage
\if@tempswa\twocolumn\fi}
\makeatother

\usepackage{amsmath,amssymb}
\usepackage{lmodern}
\usepackage{setspace}
\usepackage{iftex}
\ifPDFTeX
  \usepackage[T1]{fontenc}
  \usepackage[utf8]{inputenc}
  \usepackage{textcomp} % provide euro and other symbols
\else % if luatex or xetex
  \usepackage{unicode-math}
  \defaultfontfeatures{Scale=MatchLowercase}
  \defaultfontfeatures[\rmfamily]{Ligatures=TeX,Scale=1}
  \ifXeTeX
    \usepackage{xeCJK}
    \setCJKmainfont[]{Noto Serif CJK SC}
  \fi
  \ifLuaTeX
    \usepackage[]{luatexja-fontspec}
    \setmainjfont[]{Noto Serif CJK SC}
  \fi
\fi
% Use upquote if available, for straight quotes in verbatim environments
\IfFileExists{upquote.sty}{\usepackage{upquote}}{}
\IfFileExists{microtype.sty}{% use microtype if available
  \usepackage[]{microtype}
  \UseMicrotypeSet[protrusion]{basicmath} % disable protrusion for tt fonts
}{}
\makeatletter
\@ifundefined{KOMAClassName}{% if non-KOMA class
  \IfFileExists{parskip.sty}{%
    \usepackage{parskip}
  }{% else
    \setlength{\parindent}{0pt}
    \setlength{\parskip}{6pt plus 2pt minus 1pt}}
}{% if KOMA class
  \KOMAoptions{parskip=half}}
\makeatother
\usepackage{xcolor}
\IfFileExists{xurl.sty}{\usepackage{xurl}}{} % add URL line breaks if available
\IfFileExists{bookmark.sty}{\usepackage{bookmark}}{\usepackage{hyperref}}
\hypersetup{
  pdftitle={Readability Studio 2024.0.1},
  pdfauthor={Blake Madden},
  colorlinks=true,
  linkcolor={black},
  filecolor={Maroon},
  citecolor={black},
  urlcolor={Blue},
  pdfcreator={LaTeX via pandoc}}
\urlstyle{same} % disable monospaced font for URLs
\usepackage{color}
\usepackage{fancyvrb}
\newcommand{\VerbBar}{|}
\newcommand{\VERB}{\Verb[commandchars=\\\{\}]}
\DefineVerbatimEnvironment{Highlighting}{Verbatim}{commandchars=\\\{\}}
% Add ',fontsize=\small' for more characters per line
\usepackage{framed}
\definecolor{shadecolor}{RGB}{241,243,245}
\newenvironment{Shaded}{\begin{snugshade}}{\end{snugshade}}
\newcommand{\AlertTok}[1]{\textcolor[rgb]{0.68,0.00,0.00}{#1}}
\newcommand{\AnnotationTok}[1]{\textcolor[rgb]{0.37,0.37,0.37}{#1}}
\newcommand{\AttributeTok}[1]{\textcolor[rgb]{0.40,0.45,0.13}{#1}}
\newcommand{\BaseNTok}[1]{\textcolor[rgb]{0.68,0.00,0.00}{#1}}
\newcommand{\BuiltInTok}[1]{\textcolor[rgb]{0.00,0.23,0.31}{#1}}
\newcommand{\CharTok}[1]{\textcolor[rgb]{0.13,0.47,0.30}{#1}}
\newcommand{\CommentTok}[1]{\textcolor[rgb]{0.37,0.37,0.37}{#1}}
\newcommand{\CommentVarTok}[1]{\textcolor[rgb]{0.37,0.37,0.37}{\textit{#1}}}
\newcommand{\ConstantTok}[1]{\textcolor[rgb]{0.56,0.35,0.01}{#1}}
\newcommand{\ControlFlowTok}[1]{\textcolor[rgb]{0.00,0.23,0.31}{\textbf{#1}}}
\newcommand{\DataTypeTok}[1]{\textcolor[rgb]{0.68,0.00,0.00}{#1}}
\newcommand{\DecValTok}[1]{\textcolor[rgb]{0.68,0.00,0.00}{#1}}
\newcommand{\DocumentationTok}[1]{\textcolor[rgb]{0.37,0.37,0.37}{\textit{#1}}}
\newcommand{\ErrorTok}[1]{\textcolor[rgb]{0.68,0.00,0.00}{#1}}
\newcommand{\ExtensionTok}[1]{\textcolor[rgb]{0.00,0.23,0.31}{#1}}
\newcommand{\FloatTok}[1]{\textcolor[rgb]{0.68,0.00,0.00}{#1}}
\newcommand{\FunctionTok}[1]{\textcolor[rgb]{0.28,0.35,0.67}{#1}}
\newcommand{\ImportTok}[1]{\textcolor[rgb]{0.00,0.46,0.62}{#1}}
\newcommand{\InformationTok}[1]{\textcolor[rgb]{0.37,0.37,0.37}{#1}}
\newcommand{\KeywordTok}[1]{\textcolor[rgb]{0.00,0.23,0.31}{\textbf{#1}}}
\newcommand{\NormalTok}[1]{\textcolor[rgb]{0.00,0.23,0.31}{#1}}
\newcommand{\OperatorTok}[1]{\textcolor[rgb]{0.37,0.37,0.37}{#1}}
\newcommand{\OtherTok}[1]{\textcolor[rgb]{0.00,0.23,0.31}{#1}}
\newcommand{\PreprocessorTok}[1]{\textcolor[rgb]{0.68,0.00,0.00}{#1}}
\newcommand{\RegionMarkerTok}[1]{\textcolor[rgb]{0.00,0.23,0.31}{#1}}
\newcommand{\SpecialCharTok}[1]{\textcolor[rgb]{0.37,0.37,0.37}{#1}}
\newcommand{\SpecialStringTok}[1]{\textcolor[rgb]{0.13,0.47,0.30}{#1}}
\newcommand{\StringTok}[1]{\textcolor[rgb]{0.13,0.47,0.30}{#1}}
\newcommand{\VariableTok}[1]{\textcolor[rgb]{0.07,0.07,0.07}{#1}}
\newcommand{\VerbatimStringTok}[1]{\textcolor[rgb]{0.13,0.47,0.30}{#1}}
\newcommand{\WarningTok}[1]{\textcolor[rgb]{0.37,0.37,0.37}{\textit{#1}}}
\usepackage{longtable,booktabs,array}
\usepackage{calc} % for calculating minipage widths
% Correct order of tables after \paragraph or \subparagraph
\usepackage{etoolbox}
\makeatletter
\patchcmd\longtable{\par}{\if@noskipsec\mbox{}\fi\par}{}{}
\makeatother
% Allow footnotes in longtable head/foot
\IfFileExists{footnotehyper.sty}{\usepackage{footnotehyper}}{\usepackage{footnote}}
\makesavenoteenv{longtable}
\usepackage{graphicx}
\makeatletter
\newsavebox\pandoc@box
\newcommand*\pandocbounded[1]{% scales image to fit in text height/width
  \sbox\pandoc@box{#1}%
  \Gscale@div\@tempa{\textheight}{\dimexpr\ht\pandoc@box+\dp\pandoc@box\relax}%
  \Gscale@div\@tempb{\linewidth}{\wd\pandoc@box}%
  \ifdim\@tempb\p@<\@tempa\p@\let\@tempa\@tempb\fi% select the smaller of both
  \ifdim\@tempa\p@<\p@\scalebox{\@tempa}{\usebox\pandoc@box}%
  \else\usebox{\pandoc@box}%
  \fi%
}
% Set default figure placement to htbp
\def\fps@figure{htbp}
\makeatother
\setlength{\emergencystretch}{3em} % prevent overfull lines
\providecommand{\tightlist}{%
  \setlength{\itemsep}{0pt}\setlength{\parskip}{0pt}}
\setcounter{secnumdepth}{5}
% Make \paragraph and \subparagraph free-standing
\ifx\paragraph\undefined\else
  \let\oldparagraph\paragraph
  \renewcommand{\paragraph}[1]{\oldparagraph{#1}\mbox{}}
\fi
\ifx\subparagraph\undefined\else
  \let\oldsubparagraph\subparagraph
  \renewcommand{\subparagraph}[1]{\oldsubparagraph{#1}\mbox{}}
\fi
% LaTeX packages that we will need

% TOC
% make the space for numbers wide enough so that the section names
% don't overlap the numbers.
\makeatletter
\renewcommand{\l@section}{\@dottedtocline{1}{1.5em}{3em}}
\renewcommand{\l@subsection}{\@dottedtocline{2}{4.0em}{3.6em}}
\renewcommand{\l@subsubsection}{\@dottedtocline{3}{7.4em}{4.5em}}
\renewcommand{\l@figure}{\@dottedtocline{1}{0em}{3em}}
\let\l@table\l@figure
\makeatother

% index
\usepackage{imakeidx}
\makeindex[intoc=true,columnseprule=true,
           options= -s latex/indexstyles.ist]
\newcommand{\idxit}[1]{{\it #1}}

% to create a "see also" that appears at the bottom of the
% subentries and with no page number, do the following:
% \index{Main entry!zzzzz@\igobble|seealso{Other item}}
\def\igobble#1{}

\usepackage{changepage}
\usepackage{mdframed}
\usepackage{bbding}
\usepackage{amsthm}
\usepackage{xeCJK}
\usepackage[all]{nowidow}
\usepackage{menukeys}
\usepackage{amsmath}
\usepackage{color}
\usepackage{xcolor}
\usepackage{listings}
\usepackage[most]{tcolorbox}
\usepackage{wrapfig}
\usepackage{url}
\usepackage{setspace}
\usepackage{lettrine}
\usepackage{caption}
\usepackage{stringstrings}

% table packages
\usepackage{booktabs}
\usepackage{longtable,tabu}
\usepackage{makecell}
\usepackage{threeparttable}
\usepackage{threeparttablex}
\usepackage{colortbl}
\usepackage{float}

% fonts and graphics
\usepackage{fontawesome5}
\usepackage{fontspec}
\usepackage{adforn}
\usepackage{overpic}
\usepackage{tikz}
\usepackage{ragged2e}

% words to not hyphenate
\hyphenation{Readability}
\hyphenation{Studio}
\hyphenation{LibreOffice}
\hyphenation{Microsoft}
\hyphenation{ExportAll}
\hyphenation{FORTRAN}
\hyphenation{COBOL}
\hyphenation{FORCAST}
\hyphenation{histograms}
\hyphenation{otherwise}

% MLA requirements
\usepackage[letterpaper, margin=1in]{geometry}

% make the page headers fancy
\usepackage{fancyhdr}

\pagestyle{fancy}
\fancyhf{}
\fancyhead[LO]{\nouppercase{\leftmark}}
\fancyhead[RE]{\nouppercase{\rightmark}}
\fancyhead[LE,RO]{\thepage}

% color box that displays a quote in fancy format
% https://tex.stackexchange.com/questions/16964/block-quote-with-big-quotation-marks
\usetikzlibrary{shadows,calc,scopes,backgrounds}
\usepackage{tikzpagenodes}
\makeatletter

\tikzset{%
  fancy quotes/.style={
    text width=\fq@width pt,
    align=justify,
    inner sep=1em,
    anchor=north west,
    minimum width=\linewidth,
  },
  fancy quotes width/.initial={.8\linewidth},
  fancy quotes marks/.style={
    scale=8,
    text=white,
    inner sep=0pt,
  },
  fancy quotes opening/.style={
    fancy quotes marks,
  },
  fancy quotes closing/.style={
    fancy quotes marks,
  },
  fancy quotes background/.style={
    show background rectangle,
    inner frame xsep=0pt,
    background rectangle/.style={
      fill=gray!25,
      rounded corners,
    },
  }
}

\newenvironment{fancyquotes}[1][]{%
\noindent
\tikzpicture[fancy quotes background]
\node[fancy quotes opening,anchor=north west] (fq@ul) at (0,0) {``};
\tikz@scan@one@point\pgfutil@firstofone(fq@ul.east)
\pgfmathsetmacro{\fq@width}{\linewidth - 2*\pgf@x}
\node[fancy quotes,#1] (fq@txt) at (fq@ul.north west) \bgroup}
{\egroup;
\node[overlay,fancy quotes closing,anchor=east] at (fq@txt.south east) {''};
\endtikzpicture}

\makeatother

\makeatletter
\def\thm@space@setup{%
  \thm@preskip=8pt plus 2pt minus 4pt
  \thm@postskip=\thm@preskip
}
\makeatother
\raggedbottom

%% Provides Creative Commons Icons
\usepackage{ccicons}

% Float features
%%%%%%%%%%%%%%%%%%%%%%%%%%%%%%%%%%%%%%%%

% don't float anything
\floatplacement{table}{H}
\floatplacement{figure}{H}

% helps reduce images from floating also
\renewcommand{\topfraction}{.85}
\renewcommand{\bottomfraction}{.7}
\renewcommand{\textfraction}{.15}
\renewcommand{\floatpagefraction}{.66}
\setcounter{topnumber}{3}
\setcounter{bottomnumber}{3}
\setcounter{totalnumber}{4}

% the warning, note, and tip color boxes
%%%%%%%%%%%%%%%%%%%%%%%%%%%%%%%%%%%%%%%%

% colors that can be used
\definecolor{lightgray}{RGB}{215,215,215}
\definecolor{ultralightgray}{RGB}{241,241,241}
\definecolor{lightyellow}{RGB}{252,248,227}
\definecolor{lightpink}{RGB}{255,207,219}
\definecolor{lightblue}{RGB}{232,244,248}
\definecolor{mediumblue}{RGB}{28,77,93}

% Tip box
\newenvironment{tipsection}
    {
    \begin{tcolorbox}[colframe=lightgray,colback=lightyellow,arc=3mm]
    \faLightbulb[regular] \textbf{Tip} \newline
    }
    {
    \end{tcolorbox}
    }

% Note box
\newenvironment{notesection}
    {
    \begin{tcolorbox}[colframe=mediumblue,colback=lightblue,coltext=mediumblue,arc=3mm]
    \faEdit[regular] \textbf{Note} \newline
    }
    {
    \end{tcolorbox}
    }

% Warning box
\newenvironment{warningsection}
    {
    \begin{tcolorbox}[colframe=lightgray,colback=lightpink,arc=3mm]
    \faExclamationTriangle[solid] \textbf{{Warning} } \newline
    }
    {
    \end{tcolorbox}
    }

% A light gray box for related program options (e.g., a radio button group, a combobox and its values, etc.).
\newenvironment{optionssection}
    {
    \begin{tcolorbox}[colframe=lightgray,colback=ultralightgray,sharp corners=all,parbox=false]
    }
    {
    \end{tcolorbox}
    }

% Title to add at the top of an optionssection.
\newenvironment{optionssectiontitle}
    {
    \begin{tcolorbox}[colframe=lightgray,colback=lightgray]
    \bfseries
    }
    {
    \end{tcolorbox}
    }

%% theming
\newenvironment{darkmode}
  {
  \begin{mdframed}[backgroundcolor=black]
  \color{white}
  }
  {
  \end{mdframed}
  }

% pseudo glossary features
\newenvironment{glsentry}
  {
  \begin{minipage}{\textwidth}
  }
  {
  \end{minipage}
  }

\newenvironment{glsterm}
  {
  \bfseries
  }
  {
  }

\newenvironment{glsdef}
  {
  \noindent
  \flushleft
  \begin{adjustwidth}{2cm}{}
  }
  {
  \end{adjustwidth}
  }
\usepackage{booktabs}
\usepackage{longtable}
\usepackage{array}
\usepackage{multirow}
\usepackage{wrapfig}
\usepackage{float}
\usepackage{colortbl}
\usepackage{pdflscape}
\usepackage{tabu}
\usepackage{threeparttable}
\usepackage{threeparttablex}
\usepackage[normalem]{ulem}
\usepackage{makecell}
\usepackage{xcolor}
\makeatletter
\@ifpackageloaded{bookmark}{}{\usepackage{bookmark}}
\makeatother
\makeatletter
\@ifpackageloaded{caption}{}{\usepackage{caption}}
\AtBeginDocument{%
\ifdefined\contentsname
  \renewcommand*\contentsname{Table of contents}
\else
  \newcommand\contentsname{Table of contents}
\fi
\ifdefined\listfigurename
  \renewcommand*\listfigurename{List of Figures}
\else
  \newcommand\listfigurename{List of Figures}
\fi
\ifdefined\listtablename
  \renewcommand*\listtablename{List of Tables}
\else
  \newcommand\listtablename{List of Tables}
\fi
\ifdefined\figurename
  \renewcommand*\figurename{Figure}
\else
  \newcommand\figurename{Figure}
\fi
\ifdefined\tablename
  \renewcommand*\tablename{Table}
\else
  \newcommand\tablename{Table}
\fi
}
\@ifpackageloaded{float}{}{\usepackage{float}}
\floatstyle{ruled}
\@ifundefined{c@chapter}{\newfloat{codelisting}{h}{lop}}{\newfloat{codelisting}{h}{lop}[chapter]}
\floatname{codelisting}{Listing}
\newcommand*\listoflistings{\listof{codelisting}{List of Listings}}
\makeatother
\makeatletter
\makeatother
\makeatletter
\@ifpackageloaded{caption}{}{\usepackage{caption}}
\@ifpackageloaded{subcaption}{}{\usepackage{subcaption}}
\makeatother
\ifLuaTeX
  \usepackage{selnolig}  % disable illegal ligatures
\fi
\usepackage[style=mla,maxbibnames=4]{biblatex}
\setlength\bibitemsep{\baselineskip}
\nocite{*}

\title{Readability Studio 2024.0.1}
\usepackage{etoolbox}
\makeatletter
\providecommand{\subtitle}[1]{% add subtitle to \maketitle
  \apptocmd{\@title}{\par {\large #1 \par}}{}{}
}
\makeatother
\subtitle{System Administrator Manual}
\author{Blake Madden}
\date{2025}

\def\theauthor{Blake Madden}
\def\thedate{2025}
\def\thetitle{Readability Studio 2024.0.1}
\def\thesubtitle{System Administrator Manual}
\def\authorbio{}
\def\thepublisher{Oleander Software, Ltd.}
\def\thepublishercities{\hspace{1cm}Dayton}
\def\trademark{\textit{Readability
Studio}\textsuperscript{\tiny\textregistered}}

% Either shows a logo for the publisher or their name in bold
\def\thepublisherlogo{\large \textbf{\thepublisher}}

% Optional watermark (e.g., "DRAFT" stamped across the pages)

\begin{document}

%%%%%%%%%%%%%%%%%%%%%%%%%%%
%%%%%%%%%%%%%%%%%%%%%%%%%%%
%% Create a cover page
%%%%%%%%%%%%%%%%%%%%%%%%%%%
%%%%%%%%%%%%%%%%%%%%%%%%%%%

\thispagestyle{empty}

  \definecolor{cover-font-color}{RGB}{255, 255, 255}

  \definecolor{cover-color}{RGB}{26, 69, 135}

  \definecolor{author-color}{RGB}{61, 60, 59}

\definecolor{white-smoke}{RGB}{232, 232, 232}

\newgeometry{top=0mm, bottom=0mm, left=0mm, right=0mm}

\titlepagedecoration

% title section (at the top, overlaying image)
\tikz[overlay, remember picture] \node at (current page.north west)[anchor=north west]
  {
\begin{tcolorbox}[enhanced jigsaw, sharp corners, spread outwards, grow sidewards by=5mm, enlarge top by=-2mm,
                boxsep=0.5cm, valign=top, text height=3.75cm, boxrule=0mm,
                bicolor, colbacklower=white-smoke, halign lower=center, valign lower=center,
                    colback=cover-color, colframe=cover-color, opacityback=0.95, opacityframe=0.95]
       {\fontsize{42pt}{46pt} \usefont{T1}{PlyfrDisplay-OsF}{eb}{n} \color{cover-font-color} Readability
Studio 2024.0.1}

       \vspace{.5cm}
       {\fontsize{30pt}{34pt} \usefont{T1}{PlyfrDisplay-OsF}{eb}{it} \color{cover-font-color} System
Administrator Manual\vphantom{Qy}}

       \vspace{0.1cm}
       \tcblower
    {\large \color{black} \adforn{45}\quad {\usefont{T1}{qcs}{b}{sc} \capitalizetitle{Readability
analysis software}} \quad\adforn{46}}
  \end{tcolorbox}
};

\begin{tikzpicture}[overlay, remember picture]
    \node (centerimage) at (current page.center)[anchor=center] {\includegraphics[width=.9\linewidth]{images/non-generated/cover.png}};
\end{tikzpicture}

% author line at the bottom
\tikz[overlay, remember picture] \node at (current page.south west)[anchor=south west]
  {
  \begin{tcolorbox}[enhanced jigsaw, sharp corners, box align=bottom, halign=right, boxsep=0.5cm,
                spread outwards, grow sidewards by=5mm,
                colback=white, colframe=white, colframe=white, opacityback=1, opacityframe=1, boxrule=0mm]
    {\fontsize{18pt}{22pt} \color{author-color} {\usefont{T1}{qcs}{b}{sc} \capitalizetitle{Blake
Madden}} }
  \end{tcolorbox}
};

\begin{tikzpicture}[overlay, remember picture]
    \node (centerimage) at (current page.south west)[anchor=south west]
    {\includegraphics[width=3cm]{images/non-generated/cover-logo.png}};
\end{tikzpicture}

\clearpage{\thispagestyle{empty}\cleardoublepage}
\restoregeometry 

\frontmatter

\pagestyle{empty}

\begin{center}

    \topskip0pt
    \vspace*{\fill}

    \Huge \textsc{\textbf{\thetitle}} \par
    \LARGE \textcolor{gray}{\textit{\thesubtitle}} \par

    \vspace{1cm}

    \large \theauthor

    \vspace*{\fill}

\end{center}

\titlepagedecoration

% reset font
    \normalfont
    \normalsize

\clearpage

\clearpage

%%%%%%%%%%%%%%%%%%%%%%%%%%%
%%%%%%%%%%%%%%%%%%%%%%%%%%%
%% Create a copyright page
%%%%%%%%%%%%%%%%%%%%%%%%%%%
%%%%%%%%%%%%%%%%%%%%%%%%%%%

\input{latex/creative-commons.tex}

\clearpage

\pagestyle{fancy}

\renewcommand*\contentsname{Table of contents}
{
\hypersetup{linkcolor=}
\setcounter{tocdepth}{1}
\tableofcontents
}
\setstretch{1.15}

\mainmatter

\bookmarksetup{startatroot}

\chapter*{Preface}\label{sec-preface}
\addcontentsline{toc}{chapter}{Preface}

\markboth{Preface}{Preface}

This book is a guide to building, installing, and maintaining the
software product \emph{Readability Studio}.

System administrators can follow this guide for deploying installations
and updates of \emph{Readability Studio}. Additionally, instructions are
provided for system administrators and individual users for how to build
the program from source and optionally install it.

\emph{Readability Studio} is available for Microsoft\textsuperscript{®}
Windows \faWindows, macOS \faApple, and Linux \faLinux.

\part{Installing}

\chapter{\texorpdfstring{Microsoft\textsuperscript{®} Windows
\faWindows }{Microsoft® Windows }}\label{microsoft-windows}

\begin{longtable}[]{@{}
  >{\raggedright\arraybackslash}p{(\linewidth - 0\tabcolsep) * \real{1.0000}}@{}}
\toprule\noalign{}
\begin{minipage}[b]{\linewidth}\raggedright
Requirements
\end{minipage} \\
\midrule\noalign{}
\endhead
\bottomrule\noalign{}
\endlastfoot
64-bit Windows 10 (or higher) \\
x86\_64 processor \\
2 GB of RAM (4 GB recommended) \\
\end{longtable}

The \emph{Microsoft Visual C++ runtime} is also required. This is
normally managed via \emph{Windows Update}, but can also be installed
manually. To install, download and run the latest
\href{https://learn.microsoft.com/en-us/cpp/windows/latest-supported-vc-redist}{Microsoft
Visual C++ Redistributable} installer.

\section*{Installing}\label{installing-1}
\addcontentsline{toc}{section}{Installing}

\markright{Installing}

Run the installer (``rssetup2024.0.1.0.exe'') as an administrator. You
will be prompted for a user name, where to install the program, and
which components to install. After answering these prompts, continue the
installer to completion.

\section*{Updating}\label{updating}
\addcontentsline{toc}{section}{Updating}

\markright{Updating}

To upgrade an installation, run the installer
(``rssetup2024.0.1.0.exe'') as an administrator and follow the prompts.
If there is a previous installation of \emph{Readability Studio}, then
the installer will update it.

\begin{notesection}
If updating a 32-bit edition of \emph{Readability Studio} (version 2021
or earlier), you will first need to uninstall the program. The 64-bit
editions default to the 64-bit \emph{Windows} program folder, while
earlier editions where installed to the ``Program Files (x86)'' folder.
Uninstalling any 32-bit editions will ensure that you won't have
multiple installations after updating.

\end{notesection}

\newpage{}

\section*{Silent Mode}\label{silent-mode}
\addcontentsline{toc}{section}{Silent Mode}

\markright{Silent Mode}

The \emph{Readability Studio} installer uses \emph{Inno Setup}, which
provides support for silent install. (This applies to both installation
and upgrading.) To perform a silent install, pass the command line
arguments \texttt{/SILENT} or \texttt{/VERYSILENT}. For example, the
following will run a silent install that will not restart the computer
and write its progress to a log file:

\begin{Shaded}
\begin{Highlighting}[]
\ExtensionTok{rssetup.exe}\NormalTok{ /SILENT /NORESTART /LOG}
\end{Highlighting}
\end{Shaded}

\chapter{\texorpdfstring{macOS \faApple }{macOS }}\label{macos}

\begin{longtable}[]{@{}
  >{\raggedright\arraybackslash}p{(\linewidth - 0\tabcolsep) * \real{1.0000}}@{}}
\toprule\noalign{}
\begin{minipage}[b]{\linewidth}\raggedright
Requirements
\end{minipage} \\
\midrule\noalign{}
\endhead
\bottomrule\noalign{}
\endlastfoot
macOS 10.15 (or higher) \\
Apple Silicon or Intel processor \\
2 GB of RAM (4 GB recommended) \\
\end{longtable}

\section*{Installing}\label{installing-2}
\addcontentsline{toc}{section}{Installing}

\markright{Installing}

Open ``ReadabilityStudio.dmg'' and drag-and-drop the program into the
``Applications'' folder.

\section*{Updating}\label{updating-1}
\addcontentsline{toc}{section}{Updating}

\markright{Updating}

Open ``ReadabilityStudio.dmg'' and drag-and-drop the program into the
``Applications'' folder, replacing the existing copy.

\chapter{\texorpdfstring{Linux \faLinux }{Linux }}\label{linux}

\begin{longtable}[]{@{}
  >{\raggedright\arraybackslash}p{(\linewidth - 0\tabcolsep) * \real{1.0000}}@{}}
\toprule\noalign{}
\begin{minipage}[b]{\linewidth}\raggedright
Requirements
\end{minipage} \\
\midrule\noalign{}
\endhead
\bottomrule\noalign{}
\endlastfoot
x86\_64 processor \\
2 GB of RAM (4 GB recommended) \\
\end{longtable}

\section*{Installing}\label{installing-3}
\addcontentsline{toc}{section}{Installing}

\markright{Installing}

Make ``Readability\_Studio-x86\_64.AppImage'' executable. This can be
done with the following syntax:

\begin{Shaded}
\begin{Highlighting}[]
\NormalTok{chmod u}\SpecialCharTok{+}\NormalTok{x }\SpecialCharTok{\textless{}}\NormalTok{AppImage File}\SpecialCharTok{\textgreater{}}
\end{Highlighting}
\end{Shaded}

This can also be down by right-clicking the file, selecting
\menu[,]{{Properties},{Permissions}} and checking \emph{Allow executing
file as program}.

This only needs to be done once. Now
``Readability\_Studio-x86\_64.AppImage'' can be ran directly to launch
the program.

\section*{Updating}\label{updating-2}
\addcontentsline{toc}{section}{Updating}

\markright{Updating}

Replace your existing AppImage with the latest version. Make the new
AppImage executable and then it will be ready to run.

\part{Building from Source}

\chapter{\texorpdfstring{Microsoft\textsuperscript{®} Windows
\faWindows }{Microsoft® Windows }}\label{microsoft-windows-1}

\begin{longtable}[]{@{}
  >{\raggedright\arraybackslash}p{(\linewidth - 0\tabcolsep) * \real{1.0000}}@{}}
\toprule\noalign{}
\begin{minipage}[b]{\linewidth}\raggedright
Install the following tools to build \emph{Readability Studio}
\end{minipage} \\
\midrule\noalign{}
\endhead
\bottomrule\noalign{}
\endlastfoot
\emph{Visual Studio} \\
\emph{Inno Setup} \\
\emph{R} (optional) \\
\emph{Quarto} (optional) \\
\end{longtable}

Perform the following to build:

\section*{\texorpdfstring{Build
\emph{wxWidgets}}{Build wxWidgets}}\label{build-wxwidgets}
\addcontentsline{toc}{section}{Build \emph{wxWidgets}}

\markright{Build \emph{wxWidgets}}

\begin{itemize}
\tightlist
\item
  Open \emph{Visual Studio} and select \emph{Clone a Repository}

  \begin{itemize}
  \tightlist
  \item
    Enter
    \href{https://github.com/wxWidgets/wxWidgets.git}{``https://github.com/wxWidgets/wxWidgets.git''}
    and clone it
  \end{itemize}
\item
  Once the ``wxWidgets'' folder is cloned and opened in \emph{Visual
  Studio}:

  \begin{itemize}
  \tightlist
  \item
    Open \menu[,]{{Project},{CMake Settings for wxWidgets}}

    \begin{itemize}
    \tightlist
    \item
      Uncheck \textbf{wxBUILD\_SHARED}
    \item
      Set \textbf{wxBUILD\_OPTIMISE} to ``ON''
    \item
      Set the configuration type to ``Release''
    \item
      Save your changes
    \end{itemize}
  \item
    Select \menu[,]{{Build},{Install wxWidgets}} (builds and then copies
    the header, lib, and cmake files to the prefix folder)
  \end{itemize}
\end{itemize}

\newpage{}

\section*{\texorpdfstring{Build \emph{Readability
Studio}}{Build Readability Studio}}\label{build}
\addcontentsline{toc}{section}{Build \emph{Readability Studio}}

\markright{Build \emph{Readability Studio}}

\begin{itemize}
\tightlist
\item
  From \emph{Visual Studio}, select \emph{Clone a Repository} again

  \begin{itemize}
  \tightlist
  \item
    Enter
    \href{https://github.com/Blake-Madden/ReadabilityStudio.git}{``https://github.com/Blake-Madden/ReadabilityStudio.git''}
    and clone it to the same level as the ``wxWidgets'' folder
  \end{itemize}
\item
  Once the ``ReadabilityStudio'' folder is cloned and opened in
  \emph{Visual Studio}:

  \begin{itemize}
  \tightlist
  \item
    Open \menu[,]{{Project},{CMake Settings for readstudio}}

    \begin{itemize}
    \tightlist
    \item
      Set the configuration type to ``Release'' (or create a new release
      configuration)
    \item
      Save your changes
    \end{itemize}
  \end{itemize}
\item
  Select \menu[,]{{View},{CMake Targets}}
\item
  Build the ``readstudio'' target
\item
  Optionally, the ``manuals'' target can also be ran to rebuild the
  documentation
\end{itemize}

\section*{Build the installer}\label{build-the-installer}
\addcontentsline{toc}{section}{Build the installer}

\markright{Build the installer}

\begin{notesection}
If you build the ``Release'' version of \emph{Readability Studio}, then
the CMake build process will copy ``readstudio.exe'' into
\menu[,]{{installers},{windows},{release}}.

\end{notesection}

\begin{itemize}
\tightlist
\item
  Go to \menu[,]{{installers},{windows}}
\item
  Digitally sign ``release/readstudio.exe''
\item
  Open ``readstudio.iss'' in \emph{Inno Setup} and build the installer

  \begin{itemize}
  \tightlist
  \item
    The installer will be placed in ``output/rssetup2024.0.1.0.exe''
  \end{itemize}
\item
  Digitally sign the installer
\end{itemize}

\chapter{\texorpdfstring{macOS \faApple }{macOS }}\label{macos-1}

\begin{longtable}[]{@{}
  >{\raggedright\arraybackslash}p{(\linewidth - 0\tabcolsep) * \real{1.0000}}@{}}
\toprule\noalign{}
\begin{minipage}[b]{\linewidth}\raggedright
Install the following tools to build \emph{Readability Studio}
\end{minipage} \\
\midrule\noalign{}
\endhead
\bottomrule\noalign{}
\endlastfoot
\emph{XCode} \\
\emph{CMake} \\
\emph{create-dmg} \\
\emph{Homebrew} \\
\emph{R} (optional) \\
\emph{Quarto} (optional) \\
\end{longtable}

To install home-brew:

\begin{Shaded}
\begin{Highlighting}[]
\ExtensionTok{/bin/bash} \AttributeTok{{-}c} \DataTypeTok{\textbackslash{}}
    \StringTok{"}\VariableTok{$(}\ExtensionTok{curl} \AttributeTok{{-}fsSL}\NormalTok{ https://raw.githubusercontent.com/Homebrew/install/HEAD/install.sh}\VariableTok{)}\StringTok{"}
\end{Highlighting}
\end{Shaded}

To install \emph{CMake} and \emph{create-dmg}:

\begin{Shaded}
\begin{Highlighting}[]
\ExtensionTok{brew}\NormalTok{ install cmake}
\ExtensionTok{brew}\NormalTok{ install create{-}dmg}
\end{Highlighting}
\end{Shaded}

If you get errors about not finding a CXX compiler, run this:

\begin{Shaded}
\begin{Highlighting}[]
\FunctionTok{sudo}\NormalTok{ xcode{-}select }\AttributeTok{{-}{-}reset}
\end{Highlighting}
\end{Shaded}

If you are notarizing the app locally, then add a notarization keychain
to your system (\texttt{APPLE\_ID} should be set to your Apple account,
usually your email address). This only needs to be done once:

\begin{Shaded}
\begin{Highlighting}[]
\VariableTok{APPLE\_ID}\OperatorTok{=}
\ExtensionTok{xcrun}\NormalTok{ notarytool store{-}credentials }\AttributeTok{{-}{-}apple{-}id} \VariableTok{$\{APPLE\_ID\}}
\end{Highlighting}
\end{Shaded}

When prompted, set the keychain profile to a meaningful name and enter
your 10-digit organization ID as the team ID. Then enter your
app-specific password. (You can get that from Apple's developer
website.)

\newpage{}

\section*{\texorpdfstring{Build
\emph{wxWidgets}}{Build wxWidgets}}\label{build-wxwidgets-1}
\addcontentsline{toc}{section}{Build \emph{wxWidgets}}

\markright{Build \emph{wxWidgets}}

Download \emph{wxWidgets}:

\begin{Shaded}
\begin{Highlighting}[]
\FunctionTok{git}\NormalTok{ clone https://github.com/wxWidgets/wxWidgets.git }\AttributeTok{{-}{-}recurse{-}submodules}
\BuiltInTok{cd}\NormalTok{ wxWidgets}
\end{Highlighting}
\end{Shaded}

Before building \emph{wxWidgets}, a patch needs to be applied to add
text control features needed by the program. Follow the instructions in
\menu[,]{{wxpatch},{macOS},{Instructions.txt}} to apply the patch, and
then build:

\begin{Shaded}
\begin{Highlighting}[]
\FunctionTok{cmake}\NormalTok{ . }\AttributeTok{{-}DCMAKE\_INSTALL\_PREFIX}\OperatorTok{=}\NormalTok{./wxlib }\AttributeTok{{-}DwxBUILD\_SHARED}\OperatorTok{=}\NormalTok{OFF }\DataTypeTok{\textbackslash{}}
    \AttributeTok{{-}D}\StringTok{"CMAKE\_OSX\_ARCHITECTURES:STRING=arm64;x86\_64"} \DataTypeTok{\textbackslash{}}
    \AttributeTok{{-}DCMAKE\_OSX\_DEPLOYMENT\_TARGET}\OperatorTok{=}\NormalTok{10.13 }\DataTypeTok{\textbackslash{}}
    \AttributeTok{{-}DwxBUILD\_OPTIMISE}\OperatorTok{=}\NormalTok{ON }\AttributeTok{{-}DwxBUILD\_STRIPPED\_RELEASE}\OperatorTok{=}\NormalTok{ON }\AttributeTok{{-}DCMAKE\_BUILD\_TYPE}\OperatorTok{=}\NormalTok{Release}

\FunctionTok{cmake} \AttributeTok{{-}{-}build}\NormalTok{ . }\AttributeTok{{-}{-}target}\NormalTok{ install }\AttributeTok{{-}{-}config}\NormalTok{ Release}
\BuiltInTok{cd}\NormalTok{ ..}
\end{Highlighting}
\end{Shaded}

\section*{\texorpdfstring{Build \emph{Readability
Studio}}{Build Readability Studio}}\label{build-1}
\addcontentsline{toc}{section}{Build \emph{Readability Studio}}

\markright{Build \emph{Readability Studio}}

\begin{Shaded}
\begin{Highlighting}[]
\FunctionTok{git}\NormalTok{ clone https://github.com/Blake{-}Madden/ReadabilityStudio.git }\AttributeTok{{-}{-}recurse{-}submodules}
\BuiltInTok{cd}\NormalTok{ ReadabilityStudio}

\FunctionTok{cmake}\NormalTok{ . }\AttributeTok{{-}DCMAKE\_BUILD\_TYPE}\OperatorTok{=}\NormalTok{Release }\AttributeTok{{-}G}\NormalTok{ Xcode}
\CommentTok{\# XCode will not understand an "all" target,}
\CommentTok{\# so the binary and manuals must be built separately}
\FunctionTok{cmake} \AttributeTok{{-}{-}build}\NormalTok{ . }\AttributeTok{{-}{-}target}\NormalTok{ readstudio }\AttributeTok{{-}{-}config}\NormalTok{ Release}
\CommentTok{\# Uncomment to build the manuals}
\CommentTok{\#cmake {-}{-}build . {-}{-}target manuals}
\end{Highlighting}
\end{Shaded}

If code signing locally, do the following:

(\texttt{ORG\_ID} should be set to your 10-digit Apple developer
organization ID and \texttt{KEYCHAIN\_PROFILE} should be the keychain
profile connected to your Apple ID):

\begin{Shaded}
\begin{Highlighting}[]
\VariableTok{ORG\_ID}\OperatorTok{=}
\VariableTok{KEYCHAIN\_PROFILE}\OperatorTok{=}

\BuiltInTok{cd}\NormalTok{ installers/macos/release}

\ExtensionTok{codesign} \AttributeTok{{-}{-}force} \AttributeTok{{-}{-}verbose}\OperatorTok{=}\NormalTok{2 }\AttributeTok{{-}{-}options}\NormalTok{ runtime }\DataTypeTok{\textbackslash{}}
         \AttributeTok{{-}{-}timestamp} \AttributeTok{{-}{-}sign} \VariableTok{$\{ORG\_ID\}}\NormalTok{ ./}\StringTok{"Readability Studio.app"}

\ExtensionTok{codesign} \AttributeTok{{-}{-}verify} \AttributeTok{{-}{-}verbose}\OperatorTok{=}\NormalTok{2 ./}\StringTok{"Readability Studio.app"}
\end{Highlighting}
\end{Shaded}

\newpage{}

\section*{Build the DMG image}\label{build-the-dmg-image}
\addcontentsline{toc}{section}{Build the DMG image}

\markright{Build the DMG image}

\begin{notesection}
If you build the ``Release'' version of \emph{Readability Studio}, then
the CMake build process will copy ``Readability Studio.app'' into
\menu[,]{{installers},{macos},{release}}.

\end{notesection}

Go to \menu[,]{{installers},{macos},{release}} and run the following:

\begin{Shaded}
\begin{Highlighting}[]
\BuiltInTok{test} \AttributeTok{{-}f}\NormalTok{ ReadabilityStudio.dmg }\KeywordTok{\&\&} \FunctionTok{rm}\NormalTok{ ReadabilityStudio.dmg}
\ExtensionTok{create{-}dmg} \DataTypeTok{\textbackslash{}}
  \AttributeTok{{-}{-}volname} \StringTok{"Readability Studio Installer"} \DataTypeTok{\textbackslash{}}
  \AttributeTok{{-}{-}volicon} \StringTok{"../../app{-}logo.icns"} \DataTypeTok{\textbackslash{}}
  \AttributeTok{{-}{-}background} \StringTok{"../../app{-}logo.png"} \DataTypeTok{\textbackslash{}}
  \AttributeTok{{-}{-}window{-}pos}\NormalTok{ 200 120 }\DataTypeTok{\textbackslash{}}
  \AttributeTok{{-}{-}window{-}size}\NormalTok{ 800 400 }\DataTypeTok{\textbackslash{}}
  \AttributeTok{{-}{-}icon{-}size}\NormalTok{ 100 }\DataTypeTok{\textbackslash{}}
  \AttributeTok{{-}{-}icon} \StringTok{"readstudio.app"}\NormalTok{ 200 190 }\DataTypeTok{\textbackslash{}}
  \AttributeTok{{-}{-}hide{-}extension} \StringTok{"readstudio.app"} \DataTypeTok{\textbackslash{}}
  \AttributeTok{{-}{-}app{-}drop{-}link}\NormalTok{ 600 185 }\DataTypeTok{\textbackslash{}}
  \StringTok{"ReadabilityStudio.dmg"} \DataTypeTok{\textbackslash{}}
  \StringTok{"release/"}
\end{Highlighting}
\end{Shaded}

If notarizing locally, then sign, notarize, staple, and verify the DMG
image:

\begin{Shaded}
\begin{Highlighting}[]
\ExtensionTok{codesign} \AttributeTok{{-}{-}force} \AttributeTok{{-}{-}verbose}\OperatorTok{=}\NormalTok{2 }\AttributeTok{{-}{-}sign} \VariableTok{$\{ORG\_ID\}}\NormalTok{ ./ReadabilityStudio.dmg}
\ExtensionTok{codesign} \AttributeTok{{-}{-}verify} \AttributeTok{{-}{-}verbose}\OperatorTok{=}\NormalTok{2 ./ReadabilityStudio.dmg}
\ExtensionTok{hdiutil}\NormalTok{ verify ./ReadabilityStudio.dmg}
\ExtensionTok{xcrun}\NormalTok{ notarytool submit ./ReadabilityStudio.dmg }\DataTypeTok{\textbackslash{}}
  \AttributeTok{{-}{-}keychain{-}profile} \VariableTok{$\{KEYCHAIN\_PROFILE\}} \DataTypeTok{\textbackslash{}}
  \AttributeTok{{-}{-}wait}

\ExtensionTok{xcrun}\NormalTok{ stapler staple ./ReadabilityStudio.dmg}

\ExtensionTok{xcrun}\NormalTok{ spctl }\AttributeTok{{-}{-}assess} \AttributeTok{{-}{-}type}\NormalTok{ open }\AttributeTok{{-}{-}context}\NormalTok{ context:primary{-}signature }\DataTypeTok{\textbackslash{}}
            \AttributeTok{{-}{-}ignore{-}cache} \AttributeTok{{-}{-}verbose}\OperatorTok{=}\NormalTok{2 ./ReadabilityStudio.dmg}
\end{Highlighting}
\end{Shaded}

If you get a notarization error, run the following (\texttt{GUID} will
be the unique ID that the submission process just displayed):

\begin{Shaded}
\begin{Highlighting}[]
\ExtensionTok{xcrun}\NormalTok{ notarytool log }\VariableTok{$\{GUID\}} \AttributeTok{{-}{-}keychain{-}profile} \VariableTok{$\{KEYCHAIN\_PROFILE\}}
\end{Highlighting}
\end{Shaded}

\chapter{\texorpdfstring{Linux \faLinux }{Linux }}\label{linux-1}

\begin{longtable}[]{@{}
  >{\raggedright\arraybackslash}p{(\linewidth - 0\tabcolsep) * \real{1.0000}}@{}}
\toprule\noalign{}
\begin{minipage}[b]{\linewidth}\raggedright
Install the following tools to build \emph{Readability Studio}
\end{minipage} \\
\midrule\noalign{}
\endhead
\bottomrule\noalign{}
\endlastfoot
\emph{GCC} (C++ and fortran compilers) \\
\emph{CMake} \\
\emph{git} \\
\emph{AppImage} \\
\emph{linuxdeploy} \\
\emph{Homebrew} \\
\emph{R} (optional) \\
\emph{Quarto} (optional) \\
\end{longtable}

\begin{longtable}[]{@{}
  >{\raggedright\arraybackslash}p{(\linewidth - 0\tabcolsep) * \real{1.0000}}@{}}
\toprule\noalign{}
\begin{minipage}[b]{\linewidth}\raggedright
Install the following libraries (and their development files, if
mentioned)
\end{minipage} \\
\midrule\noalign{}
\endhead
\bottomrule\noalign{}
\endlastfoot
\emph{GTK 3}, \emph{gtk3-devel}/\emph{libgtk3-dev} \\
\emph{libCURL}, \emph{libcurl-devel}/\emph{libcurl-dev} \\
\emph{GStreamer}, \emph{gstreamer-devel} \\
\emph{libsecret}, \emph{libsecret-devel}/\emph{libsecret-1-dev} \\
\emph{webkit}, \emph{webkit2} \\
\emph{SDL2-devel} \\
\emph{libnotify}, \emph{libnotify-devel}/\emph{libnotify-dev} \\
\emph{TBB}, \emph{tbb-devel}/\emph{tbb-dev} \\
\emph{OpenMP} \\
\emph{libopenssl}, \emph{libopenssl-3-devel}/\emph{libssl-dev} \\
\emph{libxml2}, \emph{libxml2-devel}/\emph{libxml2-dev} \\
\end{longtable}

If building the documentation, prepare \emph{R} to install packages:

\begin{Shaded}
\begin{Highlighting}[]
\ExtensionTok{R}
\ExtensionTok{install.packages}\ErrorTok{(}\StringTok{"pacman"}\KeywordTok{)}
\end{Highlighting}
\end{Shaded}

\newpage{}

\section*{\texorpdfstring{Build
\emph{wxWidgets}}{Build wxWidgets}}\label{build-wxwidgets-2}
\addcontentsline{toc}{section}{Build \emph{wxWidgets}}

\markright{Build \emph{wxWidgets}}

\begin{Shaded}
\begin{Highlighting}[]
\FunctionTok{git}\NormalTok{ clone https://github.com/wxWidgets/wxWidgets.git }\AttributeTok{{-}{-}recurse{-}submodules}
\BuiltInTok{cd}\NormalTok{ wxWidgets}

\FunctionTok{cmake}\NormalTok{ . }\AttributeTok{{-}DCMAKE\_INSTALL\_PREFIX}\OperatorTok{=}\NormalTok{./wxlib }\AttributeTok{{-}DwxBUILD\_SHARED}\OperatorTok{=}\NormalTok{OFF }\DataTypeTok{\textbackslash{}}
        \AttributeTok{{-}DwxBUILD\_OPTIMISE}\OperatorTok{=}\NormalTok{ON }\AttributeTok{{-}DwxBUILD\_STRIPPED\_RELEASE}\OperatorTok{=}\NormalTok{ON }\AttributeTok{{-}DCMAKE\_BUILD\_TYPE}\OperatorTok{=}\NormalTok{Release}

\FunctionTok{cmake} \AttributeTok{{-}{-}build}\NormalTok{ . }\AttributeTok{{-}{-}target}\NormalTok{ install }\AttributeTok{{-}j} \VariableTok{$(}\FunctionTok{nproc}\VariableTok{)} \AttributeTok{{-}{-}config}\NormalTok{ Release}
\BuiltInTok{cd}\NormalTok{ ..}
\end{Highlighting}
\end{Shaded}

\section*{\texorpdfstring{Build \emph{Readability
Studio}}{Build Readability Studio}}\label{build-2}
\addcontentsline{toc}{section}{Build \emph{Readability Studio}}

\markright{Build \emph{Readability Studio}}

\begin{Shaded}
\begin{Highlighting}[]
\FunctionTok{git}\NormalTok{ clone https://github.com/Blake{-}Madden/ReadabilityStudio.git }\AttributeTok{{-}{-}recurse{-}submodules}
\BuiltInTok{cd}\NormalTok{ ReadabilityStudio}

\FunctionTok{cmake}\NormalTok{ . }\AttributeTok{{-}DCMAKE\_BUILD\_TYPE}\OperatorTok{=}\NormalTok{Release}
\CommentTok{\# Instead of "readstudio", target can also be "manuals" to build}
\CommentTok{\# the help or "all" to build the program and help}
\FunctionTok{cmake} \AttributeTok{{-}{-}build}\NormalTok{ . }\AttributeTok{{-}{-}target}\NormalTok{ readstudio }\AttributeTok{{-}j} \VariableTok{$(}\FunctionTok{nproc}\VariableTok{)} \AttributeTok{{-}{-}config}\NormalTok{ Release}
\end{Highlighting}
\end{Shaded}

\section*{Build the AppImage}\label{build-the-appimage}
\addcontentsline{toc}{section}{Build the AppImage}

\markright{Build the AppImage}

\begin{Shaded}
\begin{Highlighting}[]
\ExtensionTok{linuxdeploy{-}x86\_64.AppImage} \AttributeTok{{-}{-}appdir}\NormalTok{ installers/unix/AppDir }\DataTypeTok{\textbackslash{}}
  \AttributeTok{{-}{-}executable}\NormalTok{ installers/unix/AppDir/usr/bin/readstudio }\DataTypeTok{\textbackslash{}}
  \AttributeTok{{-}d}\NormalTok{ installers/unix/AppDir/readstudio.desktop }\DataTypeTok{\textbackslash{}}
  \AttributeTok{{-}i}\NormalTok{ installers/unix/AppDir/app{-}logo.svg }\DataTypeTok{\textbackslash{}}
  \AttributeTok{{-}{-}output}\NormalTok{ appimage}
\FunctionTok{mv}\NormalTok{ Readability\_Studio}\PreprocessorTok{*}\NormalTok{.AppImage ./installers/unix}
\end{Highlighting}
\end{Shaded}

Once built, the AppImage will be available in
\menu[,]{{installers},{unix}}. Apply a GPG signature to it before
distributing.

\backmatter

\begingroup
% 1.5 spacing just for the citation page
\setstretch{1.5}
\printbibliography[title=References,heading=bibintoc]
\endgroup


%% Index
\printindex

\clearpage{\thispagestyle{empty}\cleardoublepage}

%% Author bio at the end

\end{document}
