% LaTeX packages that we will need

% TOC
% make the space for numbers wide enough so that the section names
% don't overlap the numbers.
\makeatletter
\renewcommand{\l@section}{\@dottedtocline{1}{1.5em}{3em}}
\renewcommand{\l@subsection}{\@dottedtocline{2}{4.0em}{3.6em}}
\renewcommand{\l@subsubsection}{\@dottedtocline{3}{7.4em}{4.5em}}
\renewcommand{\l@figure}{\@dottedtocline{1}{0em}{3em}}
\let\l@table\l@figure
\makeatother

% index
\usepackage{imakeidx}
\makeindex[intoc=true,columnseprule=true,
           options= -s latex/indexstyles.ist]
\newcommand{\idxit}[1]{{\it #1}}

% to create a "see also" that appears at the bottom of the
% subentries and with no page number, do the following:
% \index{Main entry!zzzzz@\igobble|seealso{Other item}}
\def\igobble#1{}

\usepackage{changepage}
\usepackage{mdframed}
\usepackage{bbding}
\usepackage{amsthm}
\usepackage{xeCJK}
\usepackage[all]{nowidow}
\usepackage{menukeys}
\usepackage{amsmath}
\usepackage{color}
\usepackage{xcolor}
\usepackage{listings}
\usepackage[most]{tcolorbox}
\usepackage{wrapfig}
\usepackage{url}
\usepackage{setspace}
\usepackage{lettrine}
\usepackage{caption}
\usepackage{stringstrings}

% table packages
\usepackage{booktabs}
\usepackage{longtable,tabu}
\usepackage{makecell}
\usepackage{threeparttable}
\usepackage{threeparttablex}
\usepackage{colortbl}
\usepackage{float}

% fonts and graphics
\usepackage{fontawesome5}
\usepackage{fontspec}
\usepackage{adforn}
\usepackage{overpic}
\usepackage{tikz}
\usepackage{ragged2e}

% words to not hyphenate
\hyphenation{Readability}
\hyphenation{Studio}
\hyphenation{LibreOffice}
\hyphenation{Microsoft}
\hyphenation{ExportAll}
\hyphenation{FORTRAN}
\hyphenation{COBOL}
\hyphenation{FORCAST}
\hyphenation{histograms}
\hyphenation{otherwise}

% MLA requirements
\usepackage[letterpaper, margin=1in]{geometry}

% make the page headers fancy
\usepackage{fancyhdr}

\pagestyle{fancy}
\fancyhf{}
\fancyhead[LO]{\nouppercase{\leftmark}}
\fancyhead[RE]{\nouppercase{\rightmark}}
\fancyhead[LE,RO]{\thepage}

% color box that displays a quote in fancy format
% https://tex.stackexchange.com/questions/16964/block-quote-with-big-quotation-marks
\makeatletter

\tikzset{%
  fancy quotes/.style={
    text width=\fq@width pt,
    align=justify,
    inner sep=1em,
    anchor=north west,
    minimum width=\linewidth,
  },
  fancy quotes width/.initial={.8\linewidth},
  fancy quotes marks/.style={
    scale=8,
    text=white,
    inner sep=0pt,
  },
  fancy quotes opening/.style={
    fancy quotes marks,
  },
  fancy quotes closing/.style={
    fancy quotes marks,
  },
  fancy quotes background/.style={
    show background rectangle,
    inner frame xsep=0pt,
    background rectangle/.style={
      fill=gray!25,
      rounded corners,
    },
  }
}

\newenvironment{fancyquotes}[1][]{%
\noindent
\tikzpicture[fancy quotes background]
\node[fancy quotes opening,anchor=north west] (fq@ul) at (0,0) {``};
\tikz@scan@one@point\pgfutil@firstofone(fq@ul.east)
\pgfmathsetmacro{\fq@width}{\linewidth - 2*\pgf@x}
\node[fancy quotes,#1] (fq@txt) at (fq@ul.north west) \bgroup}
{\egroup;
\node[overlay,fancy quotes closing,anchor=east] at (fq@txt.south east) {''};
\endtikzpicture}

\makeatother

\makeatletter
\def\thm@space@setup{%
  \thm@preskip=8pt plus 2pt minus 4pt
  \thm@postskip=\thm@preskip
}
\makeatother
\raggedbottom

%% Provides Creative Commons Icons
\usepackage{ccicons}

% Float features
%%%%%%%%%%%%%%%%%%%%%%%%%%%%%%%%%%%%%%%%

% don't float anything
\floatplacement{table}{H}
\floatplacement{figure}{H}

% helps reduce images from floating also
\renewcommand{\topfraction}{.85}
\renewcommand{\bottomfraction}{.7}
\renewcommand{\textfraction}{.15}
\renewcommand{\floatpagefraction}{.66}
\setcounter{topnumber}{3}
\setcounter{bottomnumber}{3}
\setcounter{totalnumber}{4}

% the warning, note, and tip color boxes
%%%%%%%%%%%%%%%%%%%%%%%%%%%%%%%%%%%%%%%%

% colors that can be used
\definecolor{lightgray}{RGB}{215,215,215}
\definecolor{ultralightgray}{RGB}{241,241,241}
\definecolor{lightyellow}{RGB}{252,248,227}
\definecolor{lightpink}{RGB}{255,207,219}
\definecolor{lightblue}{RGB}{232,244,248}
\definecolor{mediumblue}{RGB}{28,77,93}

% Tip box
\newenvironment{tipsection}
    {
    \begin{tcolorbox}[colframe=lightgray,colback=lightyellow,arc=3mm]
    \faLightbulb[regular] \textbf{Tip} \newline
    }
    {
    \end{tcolorbox}
    }

% Note box
\newenvironment{notesection}
    {
    \begin{tcolorbox}[colframe=mediumblue,colback=lightblue,coltext=mediumblue,arc=3mm]
    \faEdit[regular] \textbf{Note} \newline
    }
    {
    \end{tcolorbox}
    }

% Warning box
\newenvironment{warningsection}
    {
    \begin{tcolorbox}[colframe=lightgray,colback=lightpink,arc=3mm]
    \faExclamationTriangle[solid] \textbf{{Warning} } \newline
    }
    {
    \end{tcolorbox}
    }

% A light gray box for related program options (e.g., a radio button group, a combobox and its values, etc.).
\newenvironment{optionssection}
    {
    \begin{tcolorbox}[colframe=lightgray,colback=ultralightgray,sharp corners=all,parbox=false]
    }
    {
    \end{tcolorbox}
    }

% Title to add at the top of an optionssection.
\newenvironment{optionssectiontitle}
    {
    \begin{tcolorbox}[colframe=lightgray,colback=lightgray]
    \bfseries
    }
    {
    \end{tcolorbox}
    }

%% theming
\newenvironment{darkmode}
  {
  \begin{mdframed}[backgroundcolor=black]
  \color{white}
  }
  {
  \end{mdframed}
  }

% pseudo glossary features
\newenvironment{glsentry}
  {
  \begin{minipage}{\textwidth}
  }
  {
  \end{minipage}
  }

\newenvironment{glsterm}
  {
  \bfseries
  }
  {
  }

\newenvironment{glsdef}
  {
  \noindent
  \flushleft
  \begin{adjustwidth}{2cm}{}
  }
  {
  \end{adjustwidth}
  }
