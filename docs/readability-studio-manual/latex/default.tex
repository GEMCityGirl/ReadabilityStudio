% Options for packages loaded elsewhere
\PassOptionsToPackage{unicode$for(hyperrefoptions)$,$hyperrefoptions$$endfor$}{hyperref}
\PassOptionsToPackage{hyphens}{url}
$if(colorlinks)$
\PassOptionsToPackage{dvipsnames,svgnames,x11names}{xcolor}
$endif$
$if(dir)$
$if(latex-dir-rtl)$
\PassOptionsToPackage{RTLdocument}{bidi}
$endif$
$endif$
$if(CJKmainfont)$
\PassOptionsToPackage{space}{xeCJK}
$endif$
%
\documentclass[
$if(fontsize)$
  $fontsize$,
$endif$
$if(lang)$
  $babel-lang$,
$endif$
$if(papersize)$
  $papersize$paper,
$endif$
$if(beamer)$
  ignorenonframetext,
$if(handout)$
  handout,
$endif$
$if(aspectratio)$
  aspectratio=$aspectratio$,
$endif$
$endif$
$for(classoption)$
  $classoption$$sep$,
$endfor$
]{$documentclass$}

%%%%%%%%%%%%%%%%%%%%%%%%%%%%%%%%%%%%
%
% Settings for Legrand Orange Book document class
%
%%%%%%%%%%%%%%%%%%%%%%%%%%%%%%%%%%%%

$if(theme-color)$
  \definecolor{ocre}{RGB}{$theme-color$}
$else$
  \definecolor{ocre}{RGB}{243, 102, 25} % Define the color used for highlighting throughout the book
$endif$

$if(chapter-image)$
  \chapterimage{$chapter-image$} % Chapter heading image
$endif$
\chapterspaceabove{6.5cm} % Default whitespace from the top of the page to the chapter title on chapter pages
\chapterspacebelow{6.75cm} % Default amount of vertical whitespace from the top margin to the start of the text on chapter pages

% title page

\colorlet{titlepagecolor}{ocre}

% The following code is borrowed from: https://tex.stackexchange.com/a/86310/10898

\newcommand\titlepagedecoration{%
\begin{tikzpicture}[remember picture,overlay,shorten >= -10pt]

\coordinate (aux1) at ([yshift=-15pt]current page.north east);
\coordinate (aux2) at ([yshift=-410pt]current page.north east);
\coordinate (aux3) at ([xshift=-4.5cm]current page.north east);
\coordinate (aux4) at ([yshift=-150pt]current page.north east);

\begin{scope}[titlepagecolor!40,line width=12pt,rounded corners=12pt]
\draw
  (aux1) -- coordinate (a)
  ++(225:5) --
  ++(-45:5.1) coordinate (b);
\draw[shorten <= -10pt]
  (aux3) --
  (a) --
  (aux1);
\draw[opacity=0.6,titlepagecolor,shorten <= -10pt]
  (b) --
  ++(225:2.2) --
  ++(-45:2.2);
\end{scope}
\draw[titlepagecolor,line width=8pt,rounded corners=8pt,shorten <= -10pt]
  (aux4) --
  ++(225:0.8) --
  ++(-45:0.8);
\begin{scope}[titlepagecolor!70,line width=6pt,rounded corners=8pt]
\draw[shorten <= -10pt]
  (aux2) --
  ++(225:3) coordinate[pos=0.45] (c) --
  ++(-45:3.1);
\draw
  (aux2) --
  (c) --
  ++(135:2.5) --
  ++(45:2.5) --
  ++(-45:2.5) coordinate[pos=0.3] (d);   
\draw 
  (d) -- +(45:1);
\end{scope}
\end{tikzpicture}%
}

% add the title page decoration
\makeatletter
\def\@endpart{\titlepagedecoration\vfil\newpage
\if@tempswa\twocolumn\fi}
\makeatother

$if(beamer)$
$if(background-image)$
\usebackgroundtemplate{%
  \includegraphics[width=\paperwidth]{$background-image$}%
}
$endif$
\usepackage{pgfpages}
\setbeamertemplate{caption}[numbered]
\setbeamertemplate{caption label separator}{: }
\setbeamercolor{caption name}{fg=normal text.fg}
\beamertemplatenavigationsymbols$if(navigation)$$navigation$$else$empty$endif$
$for(beameroption)$
\setbeameroption{$beameroption$}
$endfor$
% Prevent slide breaks in the middle of a paragraph
\widowpenalties 1 10000
\raggedbottom
$if(section-titles)$
\setbeamertemplate{part page}{
  \centering
  \begin{beamercolorbox}[sep=16pt,center]{part title}
    \usebeamerfont{part title}\insertpart\par
  \end{beamercolorbox}
}
\setbeamertemplate{section page}{
  \centering
  \begin{beamercolorbox}[sep=12pt,center]{part title}
    \usebeamerfont{section title}\insertsection\par
  \end{beamercolorbox}
}
\setbeamertemplate{subsection page}{
  \centering
  \begin{beamercolorbox}[sep=8pt,center]{part title}
    \usebeamerfont{subsection title}\insertsubsection\par
  \end{beamercolorbox}
}
\AtBeginPart{
  \frame{\partpage}
}
\AtBeginSection{
  \ifbibliography
  \else
    \frame{\sectionpage}
  \fi
}
\AtBeginSubsection{
  \frame{\subsectionpage}
}
$endif$
$endif$
$if(beamerarticle)$
\usepackage{beamerarticle} % needs to be loaded first
$endif$
\usepackage{amsmath,amssymb}
$if(fontfamily)$
\usepackage[$for(fontfamilyoptions)$$fontfamilyoptions$$sep$,$endfor$]{$fontfamily$}
$else$
\usepackage{lmodern}
$endif$
$if(linestretch)$
\usepackage{setspace}
$endif$
\usepackage{iftex}
\ifPDFTeX
  \usepackage[$if(fontenc)$$fontenc$$else$T1$endif$]{fontenc}
  \usepackage[utf8]{inputenc}
  \usepackage{textcomp} % provide euro and other symbols
\else % if luatex or xetex
$if(mathspec)$
  \ifXeTeX
    \usepackage{mathspec}
  \else
    \usepackage{unicode-math}
  \fi
$else$
  \usepackage{unicode-math}
$endif$
  \defaultfontfeatures{Scale=MatchLowercase}
  \defaultfontfeatures[\rmfamily]{Ligatures=TeX,Scale=1}
$if(mainfont)$
  \setmainfont[$for(mainfontoptions)$$mainfontoptions$$sep$,$endfor$]{$mainfont$}
$endif$
$if(sansfont)$
  \setsansfont[$for(sansfontoptions)$$sansfontoptions$$sep$,$endfor$]{$sansfont$}
$endif$
$if(monofont)$
  \setmonofont[$for(monofontoptions)$$monofontoptions$$sep$,$endfor$]{$monofont$}
$endif$
$for(fontfamilies)$
  \newfontfamily{$fontfamilies.name$}[$for(fontfamilies.options)$$fontfamilies.options$$sep$,$endfor$]{$fontfamilies.font$}
$endfor$
$if(mathfont)$
$if(mathspec)$
  \ifXeTeX
    \setmathfont(Digits,Latin,Greek)[$for(mathfontoptions)$$mathfontoptions$$sep$,$endfor$]{$mathfont$}
  \else
    \setmathfont[$for(mathfontoptions)$$mathfontoptions$$sep$,$endfor$]{$mathfont$}
  \fi
$else$
  \setmathfont[$for(mathfontoptions)$$mathfontoptions$$sep$,$endfor$]{$mathfont$}
$endif$
$endif$
$if(CJKmainfont)$
  \ifXeTeX
    \usepackage{xeCJK}
    \setCJKmainfont[$for(CJKoptions)$$CJKoptions$$sep$,$endfor$]{$CJKmainfont$}
  \fi
$endif$
$if(luatexjapresetoptions)$
  \ifLuaTeX
    \usepackage[$for(luatexjapresetoptions)$$luatexjapresetoptions$$sep$,$endfor$]{luatexja-preset}
  \fi
$endif$
$if(CJKmainfont)$
  \ifLuaTeX
    \usepackage[$for(luatexjafontspecoptions)$$luatexjafontspecoptions$$sep$,$endfor$]{luatexja-fontspec}
    \setmainjfont[$for(CJKoptions)$$CJKoptions$$sep$,$endfor$]{$CJKmainfont$}
  \fi
$endif$
\fi
$if(zero-width-non-joiner)$
%% Support for zero-width non-joiner characters.
\makeatletter
\def\zerowidthnonjoiner{%
  % Prevent ligatures and adjust kerning, but still support hyphenating.
  \texorpdfstring{%
    \textormath{\nobreak\discretionary{-}{}{\kern.03em}%
      \ifvmode\else\nobreak\hskip\z@skip\fi}{}%
  }{}%
}
\makeatother
\ifPDFTeX
  \DeclareUnicodeCharacter{200C}{\zerowidthnonjoiner}
\else
  \catcode`^^^^200c=\active
  \protected\def ^^^^200c{\zerowidthnonjoiner}
\fi
%% End of ZWNJ support
$endif$
$if(beamer)$
$if(theme)$
\usetheme[$for(themeoptions)$$themeoptions$$sep$,$endfor$]{$theme$}
$endif$
$if(colortheme)$
\usecolortheme{$colortheme$}
$endif$
$if(fonttheme)$
\usefonttheme{$fonttheme$}
$endif$
$if(mainfont)$
\usefonttheme{serif} % use mainfont rather than sansfont for slide text
$endif$
$if(innertheme)$
\useinnertheme{$innertheme$}
$endif$
$if(outertheme)$
\useoutertheme{$outertheme$}
$endif$
$endif$
% Use upquote if available, for straight quotes in verbatim environments
\IfFileExists{upquote.sty}{\usepackage{upquote}}{}
\IfFileExists{microtype.sty}{% use microtype if available
  \usepackage[$for(microtypeoptions)$$microtypeoptions$$sep$,$endfor$]{microtype}
  \UseMicrotypeSet[protrusion]{basicmath} % disable protrusion for tt fonts
}{}
$if(indent)$
$else$
\makeatletter
\@ifundefined{KOMAClassName}{% if non-KOMA class
  \IfFileExists{parskip.sty}{%
    \usepackage{parskip}
  }{% else
    \setlength{\parindent}{0pt}
    \setlength{\parskip}{6pt plus 2pt minus 1pt}}
}{% if KOMA class
  \KOMAoptions{parskip=half}}
\makeatother
$endif$
$if(verbatim-in-note)$
\usepackage{fancyvrb}
$endif$
\usepackage{xcolor}
\IfFileExists{xurl.sty}{\usepackage{xurl}}{} % add URL line breaks if available
\IfFileExists{bookmark.sty}{\usepackage{bookmark}}{\usepackage{hyperref}}
\hypersetup{
$if(title-meta)$
  pdftitle={$title-meta$},
$endif$
$if(author-meta)$
  pdfauthor={$author-meta$},
$endif$
$if(lang)$
  pdflang={$lang$},
$endif$
$if(subject)$
  pdfsubject={$subject$},
$endif$
$if(keywords)$
  pdfkeywords={$for(keywords)$$keywords$$sep$, $endfor$},
$endif$
$if(colorlinks)$
  colorlinks=true,
  linkcolor={$if(linkcolor)$$linkcolor$$else$Maroon$endif$},
  filecolor={$if(filecolor)$$filecolor$$else$Maroon$endif$},
  citecolor={$if(citecolor)$$citecolor$$else$Blue$endif$},
  urlcolor={$if(urlcolor)$$urlcolor$$else$Blue$endif$},
$else$
  hidelinks,
$endif$
  pdfcreator={LaTeX via pandoc}}
\urlstyle{same} % disable monospaced font for URLs
$if(verbatim-in-note)$
\VerbatimFootnotes % allow verbatim text in footnotes
$endif$
$if(geometry)$
$if(beamer)$
\geometry{$for(geometry)$$geometry$$sep$,$endfor$}
$else$
\usepackage[$for(geometry)$$geometry$$sep$,$endfor$]{geometry}
$endif$
$endif$
$if(beamer)$
\newif\ifbibliography
$endif$
$if(listings)$
\usepackage{listings}
\newcommand{\passthrough}[1]{#1}
\lstset{defaultdialect=[5.3]Lua}
\lstset{defaultdialect=[x86masm]Assembler}
$endif$
$if(lhs)$
\lstnewenvironment{code}{\lstset{language=Haskell,basicstyle=\small\ttfamily}}{}
$endif$
$if(highlighting-macros)$
$highlighting-macros$
$endif$
$if(tables)$
\usepackage{longtable,booktabs,array}
$if(multirow)$
\usepackage{multirow}
$endif$
\usepackage{calc} % for calculating minipage widths
$if(beamer)$
\usepackage{caption}
% Make caption package work with longtable
\makeatletter
\def\fnum@table{\tablename~\thetable}
\makeatother
$else$
% Correct order of tables after \paragraph or \subparagraph
\usepackage{etoolbox}
\makeatletter
\patchcmd\longtable{\par}{\if@noskipsec\mbox{}\fi\par}{}{}
\makeatother
% Allow footnotes in longtable head/foot
\IfFileExists{footnotehyper.sty}{\usepackage{footnotehyper}}{\usepackage{footnote}}
\makesavenoteenv{longtable}
$endif$
$endif$
$if(graphics)$
\usepackage{graphicx}
\makeatletter
\def\maxwidth{\ifdim\Gin@nat@width>\linewidth\linewidth\else\Gin@nat@width\fi}
\def\maxheight{\ifdim\Gin@nat@height>\textheight\textheight\else\Gin@nat@height\fi}
\makeatother
% Scale images if necessary, so that they will not overflow the page
% margins by default, and it is still possible to overwrite the defaults
% using explicit options in \includegraphics[width, height, ...]{}
\setkeys{Gin}{width=\maxwidth,height=\maxheight,keepaspectratio}
% Set default figure placement to htbp
\makeatletter
\def\fps@figure{htbp}
\makeatother
$endif$
$if(links-as-notes)$
% Make links footnotes instead of hotlinks:
\DeclareRobustCommand{\href}[2]{#2\footnote{\url{#1}}}
$endif$
$if(strikeout)$
\usepackage[normalem]{ulem}
% Avoid problems with \sout in headers with hyperref
\pdfstringdefDisableCommands{\renewcommand{\sout}{}}
$endif$
\setlength{\emergencystretch}{3em} % prevent overfull lines
\providecommand{\tightlist}{%
  \setlength{\itemsep}{0pt}\setlength{\parskip}{0pt}}
$if(numbersections)$
\setcounter{secnumdepth}{$if(secnumdepth)$$secnumdepth$$else$5$endif$}
$else$
\setcounter{secnumdepth}{-\maxdimen} % remove section numbering
$endif$
$if(beamer)$
$else$
$if(block-headings)$
% Make \paragraph and \subparagraph free-standing
\ifx\paragraph\undefined\else
  \let\oldparagraph\paragraph
  \renewcommand{\paragraph}[1]{\oldparagraph{#1}\mbox{}}
\fi
\ifx\subparagraph\undefined\else
  \let\oldsubparagraph\subparagraph
  \renewcommand{\subparagraph}[1]{\oldsubparagraph{#1}\mbox{}}
\fi
$endif$
$endif$
$if(pagestyle)$
\pagestyle{$pagestyle$}
$endif$
$if(csl-refs)$
% definitions for citeproc citations
\NewDocumentCommand\citeproctext{}{}
\NewDocumentCommand\citeproc{mm}{%
\begingroup\def\citeproctext{#2}\cite{#1}\endgroup}
\makeatletter
% allow citations to break across lines
\let\@cite@ofmt\@firstofone
% avoid brackets around text for \cite:
\def\@biblabel#1{}
\def\@cite#1#2{{#1\if@tempswa , #2\fi}}
\makeatother
\newlength{\cslhangindent}
\setlength{\cslhangindent}{1.5em}
\newlength{\csllabelwidth}
\setlength{\csllabelwidth}{3em}
\newenvironment{CSLReferences}[2] % #1 hanging-indent, #2 entry-spacing
{\begin{list}{}{%
	\setlength{\itemindent}{0pt}
	\setlength{\leftmargin}{0pt}
	\setlength{\parsep}{0pt}
	% turn on hanging indent if param 1 is 1
	\ifodd #1
	\setlength{\leftmargin}{\cslhangindent}
	\setlength{\itemindent}{-1\cslhangindent}
	\fi
	% set entry spacing
	\setlength{\itemsep}{#2\baselineskip}}}
{\end{list}}
\usepackage{calc}
\newcommand{\CSLBlock}[1]{\hfill\break\parbox[t]{\linewidth}{\strut\ignorespaces#1\strut}}
\newcommand{\CSLLeftMargin}[1]{\parbox[t]{\csllabelwidth}{\strut#1\strut}}
\newcommand{\CSLRightInline}[1]{\parbox[t]{\linewidth - \csllabelwidth}{\strut#1\strut}}
\newcommand{\CSLIndent}[1]{\hspace{\cslhangindent}#1}
$endif$
$for(header-includes)$
$header-includes$
$endfor$
$if(lang)$
\ifXeTeX
  % Load polyglossia as late as possible: uses bidi with RTL langages (e.g. Hebrew, Arabic)
  \usepackage{polyglossia}
  \setmainlanguage[$for(polyglossia-lang.options)$$polyglossia-lang.options$$sep$,$endfor$]{$polyglossia-lang.name$}
$for(polyglossia-otherlangs)$
  \setotherlanguage[$for(polyglossia-otherlangs.options)$$polyglossia-otherlangs.options$$sep$,$endfor$]{$polyglossia-otherlangs.name$}
$endfor$
\else
  \usepackage[$for(babel-otherlangs)$$babel-otherlangs$,$endfor$main=$babel-lang$]{babel}
% get rid of language-specific shorthands (see #6817):
\let\LanguageShortHands\languageshorthands
\def\languageshorthands#1{}
$if(babel-newcommands)$
  $babel-newcommands$
$endif$
\fi
$endif$
\ifLuaTeX
  \usepackage{selnolig}  % disable illegal ligatures
\fi
$if(dir)$
\ifXeTeX
  % Load bidi as late as possible as it modifies e.g. graphicx
  \usepackage{bidi}
\fi
\ifPDFTeX
  \TeXXeTstate=1
  \newcommand{\RL}[1]{\beginR #1\endR}
  \newcommand{\LR}[1]{\beginL #1\endL}
  \newenvironment{RTL}{\beginR}{\endR}
  \newenvironment{LTR}{\beginL}{\endL}
\fi
$endif$
$if(natbib)$
\usepackage[$natbiboptions$]{natbib}
\bibliographystyle{$if(biblio-style)$$biblio-style$$else$plainnat$endif$}
$endif$
$if(biblatex)$
\usepackage[$if(biblio-style)$style=$biblio-style$,$endif$$for(biblatexoptions)$$biblatexoptions$$sep$,$endfor$]{biblatex}
\setlength\bibitemsep{\baselineskip}
$for(bibliography)$
\addbibresource{$bibliography$}
$endfor$
$endif$
$if(nocite-ids)$
\nocite{$for(nocite-ids)$$it$$sep$, $endfor$}
$endif$
$if(csquotes)$
\usepackage{csquotes}
$endif$

$if(title)$
\title{$title$$if(thanks)$\thanks{$thanks$}$endif$}
$endif$
$if(subtitle)$
$if(beamer)$
$else$
\usepackage{etoolbox}
\makeatletter
\providecommand{\subtitle}[1]{% add subtitle to \maketitle
  \apptocmd{\@title}{\par {\large #1 \par}}{}{}
}
\makeatother
$endif$
\subtitle{$subtitle$}
$endif$
\author{$for(author)$$author$$sep$ \and $endfor$}
\date{$date$}
$if(beamer)$
$if(institute)$
\institute{$for(institute)$$institute$$sep$ \and $endfor$}
$endif$
$if(titlegraphic)$
\titlegraphic{\includegraphics{$titlegraphic$}}
$endif$
$if(logo)$
\logo{\includegraphics{$logo$}}
$endif$
$endif$

\def\theauthor{$for(author)$$author$$endfor$}
\def\thedate{$for(date)$$date$$endfor$}
\def\thetitle{$for(title)$$title$$endfor$}
\def\thesubtitle{$for(subtitle)$$subtitle$$endfor$}
\def\authorbio{$for(authorbio)$$authorbio$$endfor$}

% Optional watermark (e.g., "DRAFT" stamped across the pages)
$if(watermark)$
\usepackage[text=$watermark$]{draftwatermark}
$endif$

\begin{document}

%%%%%%%%%%%%%%%%%%%%%%%%%%%
%%%%%%%%%%%%%%%%%%%%%%%%%%%
%% Create a cover page
%%%%%%%%%%%%%%%%%%%%%%%%%%%
%%%%%%%%%%%%%%%%%%%%%%%%%%%

$if(pdf-cover)$
\thispagestyle{empty}

$if(pdf-cover-font-color)$
  \definecolor{cover-font-color}{RGB}{$pdf-cover-font-color$}
$else$
  \definecolor{cover-font-color}{RGB}{0, 0, 0}
$endif$

$if(pdf-cover-color)$
  \definecolor{cover-color}{RGB}{$pdf-cover-color$}
$else$
  \colorlet{cover-color}{ocre}
$endif$

\definecolor{white-smoke}{RGB}{232, 232, 232}

\newgeometry{top=0mm, bottom=0mm, left=0mm, right=0mm}

% draw the image with a shadow underneath it
\newsavebox\myshadowbox
\newlength\myshadowlen

\newcommand\shadowimage[2][]
  {%
    {[on background layer]
    \setbox0=\hbox{\includegraphics[#1]{#2}}

    \begin{tikzpicture}
    \node[anchor=south west, inner sep=0, outer sep=0] (image) at (0, 0) {\includegraphics[#1]{#2}};

    \setlength\myshadowlen{\wd0}
    \ifnum\myshadowlen<\ht0
      \setlength\myshadowlen{\ht0}
    \fi
    \divide \myshadowlen by 120

    \drawshadow{image}
    
    \end{tikzpicture}
    }
  }

\noindent\shadowimage[width=.9\linewidth]{$pdf-cover$}

% title section (at the top, overlaying image)
\tikz[overlay, remember picture] \node at (current page.north west)[anchor=north west]
  {
\begin{tcolorbox}[enhanced jigsaw, sharp corners, spread outwards, grow sidewards by=5mm, enlarge top by=-2mm,
                boxsep=0.5cm, valign=top, text height=3.75cm, boxrule=0mm,
                bicolor, colbacklower=white-smoke, halign lower=center, valign lower=center,
                    colback=cover-color, colframe=cover-color, opacityback=0.95, opacityframe=0.95]
       {\fontsize{42pt}{46pt} \usefont{T1}{PlyfrDisplay-OsF}{eb}{n} \color{cover-font-color} $title$}

       \vspace{.5cm}
       {\fontsize{30pt}{34pt} \usefont{T1}{PlyfrDisplay-OsF}{eb}{it} \color{cover-font-color} $subtitle$\vphantom{Qy}}

       \vspace{0.1cm}
       \tcblower
    {\large \color{black} \adforn{45}\quad {\usefont{T1}{qcs}{b}{sc} \capitalizetitle{$description$}} \quad\adforn{46}}
  \end{tcolorbox}
};

% author line at the bottom
\tikz[overlay, remember picture] \node at (current page.south west)[anchor=south west]
  {
  \begin{tcolorbox}[enhanced jigsaw, sharp corners, box align=bottom, halign=right, boxsep=0.5cm,
                spread outwards, grow sidewards by=5mm,
                colback=white, colframe=white, colframe=white, opacityback=1, opacityframe=1, boxrule=0mm]
    {\fontsize{18pt}{22pt} \color{cover-color} {\usefont{T1}{qcs}{b}{sc} \capitalizetitle{$author$}} }
  \end{tcolorbox}
};

\clearpage{\thispagestyle{empty}\cleardoublepage}
\restoregeometry 
$endif$

\frontmatter

\pagestyle{empty}

\begin{center}

    \topskip0pt
    \vspace*{\fill}

    \Huge \textsc{\textbf{\thetitle}} \par
    \LARGE \textit{\thesubtitle} \par

    \vspace{1cm}

    \large \theauthor

    \vspace*{\fill}

\end{center}

% reset font
    \normalfont
    \normalsize

\clearpage

\clearpage

%%%%%%%%%%%%%%%%%%%%%%%%%%%
%%%%%%%%%%%%%%%%%%%%%%%%%%%
%% Create a copyright page
%%%%%%%%%%%%%%%%%%%%%%%%%%%
%%%%%%%%%%%%%%%%%%%%%%%%%%%

\pagestyle{empty}

% Center everything
  \begin{center}

% Use Sans font
  \sffamily

% Write the Book copyright page
  \makeatletter  \small \thetitle \par \makeatother
  \makeatletter  \small \textit{\thesubtitle} \par \makeatother

% Write the CC logo, year, and author
    Copyright \ccLogo\ \makeatletter \thedate\ \theauthor \makeatother \par
    \trademark\ is a trademark of \thepublisher \par
    All rights reserved. \par

% Add a vertical space
    \vspace{0.4cm}

% Write what country it was published in
    Published in the United States

% Add a vertical space
    \vspace{0.4cm}

% Write the specific license name
    This book is distributed under a Creative Commons Attribution-Sharealike 4.0 License. \par

% Add a vertical space
    \vspace{0.4cm}

% Write the Creative Commons Icons
    \ccbysa

% Stop centering everything
  \end{center}

% Write license text
   \scriptsize
   \noindent
    That means you are free:
      \begin{itemize}
        \setlength{\itemsep}{0pt}
        \setlength{\parskip}{0pt}
        \setlength{\parsep}{0pt} 
         \item \textbf{To Share} -- copy and redistribute the material in any medium or format.
         \item \textbf{To Adapt} -- remix, transform, and build upon the material.
      \end{itemize}
    The licensor cannot revoke these freedoms as long as you follow the license terms: \par
      \begin{itemize}
        \setlength{\itemsep}{0pt}
        \setlength{\parskip}{0pt}
        \setlength{\parsep}{0pt}
          \item \textbf{Attribution} -- You must give appropriate credit, provide a link to the license, and indicate if changes were made. You may do so in any reasonable manner, but not in any way that suggests the licensor endorses you or your use. \par
          \item \textbf{Share Alike} -- If you remix, transform, or build upon the material, you must distribute your contributions under the same license as the original. \par
      \end{itemize}
    \textbf{No additional restrictions — You may not apply legal terms or technological measures that legally restrict others from doing anything the license permits.}

% reset font
    \normalfont
    \normalsize

% End the page
\clearpage

\clearpage

\pagestyle{fancy}

$for(include-before)$
$include-before$

$endfor$
$if(toc)$
$if(toc-title)$
\renewcommand*\contentsname{$toc-title$}
$endif$
$if(beamer)$
\begin{frame}[allowframebreaks]
$if(toc-title)$
  \frametitle{$toc-title$}
$endif$
  \tableofcontents[hideallsubsections]
\end{frame}
$else$
{
$if(colorlinks)$
\hypersetup{linkcolor=$if(toccolor)$$toccolor$$else$$endif$}
$endif$
\setcounter{tocdepth}{$toc-depth$}
\tableofcontents
}
$endif$
$endif$
$if(lof)$
\listoffigures
$endif$
$if(lot)$
\listoftables
$endif$
$if(linestretch)$
\setstretch{$linestretch$}
$endif$

\mainmatter

$body$

\backmatter

$if(natbib)$
$if(bibliography)$
$if(biblio-title)$
$if(has-chapters)$
\renewcommand\bibname{$biblio-title$}
$else$
\renewcommand\refname{$biblio-title$}
$endif$
$endif$
$if(beamer)$
\begin{frame}[allowframebreaks]{$biblio-title$}
  \bibliographytrue
$endif$
  \bibliography{$for(bibliography)$$bibliography$$sep$,$endfor$}
$if(beamer)$
\end{frame}
$endif$

$endif$
$endif$
$if(biblatex)$
$if(beamer)$
\begin{frame}[allowframebreaks]{$biblio-title$}
  \bibliographytrue
  \printbibliography[heading=none]
\end{frame}
$else$
\begingroup
% 1.5 spacing just for the citation page
\setstretch{1.5}
\printbibliography$if(biblio-title)$[title=$biblio-title$,heading=bibintoc]$endif$
\endgroup
$endif$

$endif$
$for(include-after)$
$include-after$

$endfor$

%% Index
\printindex

\clearpage{\thispagestyle{empty}\cleardoublepage}

%% Author bio at the end
$if(authorbio)$
\pagestyle{empty}
\chapter*{About the Author}
\addcontentsline{toc}{chapter}{About the Author}
\label{about-the-author}

\authorbio
$endif$

\end{document}